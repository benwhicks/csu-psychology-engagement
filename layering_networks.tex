\PassOptionsToPackage{unicode=true}{hyperref} % options for packages loaded elsewhere
\PassOptionsToPackage{hyphens}{url}
%
\documentclass[
  ignorenonframetext,
]{beamer}
\usepackage{pgfpages}
\setbeamertemplate{caption}[numbered]
\setbeamertemplate{caption label separator}{: }
\setbeamercolor{caption name}{fg=normal text.fg}
\beamertemplatenavigationsymbolsempty
% Prevent slide breaks in the middle of a paragraph:
\widowpenalties 1 10000
\raggedbottom
\setbeamertemplate{part page}{
  \centering
  \begin{beamercolorbox}[sep=16pt,center]{part title}
    \usebeamerfont{part title}\insertpart\par
  \end{beamercolorbox}
}
\setbeamertemplate{section page}{
  \centering
  \begin{beamercolorbox}[sep=12pt,center]{part title}
    \usebeamerfont{section title}\insertsection\par
  \end{beamercolorbox}
}
\setbeamertemplate{subsection page}{
  \centering
  \begin{beamercolorbox}[sep=8pt,center]{part title}
    \usebeamerfont{subsection title}\insertsubsection\par
  \end{beamercolorbox}
}
\AtBeginPart{
  \frame{\partpage}
}
\AtBeginSection{
  \ifbibliography
  \else
    \frame{\sectionpage}
  \fi
}
\AtBeginSubsection{
  \frame{\subsectionpage}
}
\usepackage{lmodern}
\usepackage{amssymb,amsmath}
\usepackage{ifxetex,ifluatex}
\ifnum 0\ifxetex 1\fi\ifluatex 1\fi=0 % if pdftex
  \usepackage[T1]{fontenc}
  \usepackage[utf8]{inputenc}
  \usepackage{textcomp} % provides euro and other symbols
\else % if luatex or xelatex
  \usepackage{unicode-math}
  \defaultfontfeatures{Scale=MatchLowercase}
  \defaultfontfeatures[\rmfamily]{Ligatures=TeX,Scale=1}
\fi
% use upquote if available, for straight quotes in verbatim environments
\IfFileExists{upquote.sty}{\usepackage{upquote}}{}
\IfFileExists{microtype.sty}{% use microtype if available
  \usepackage[]{microtype}
  \UseMicrotypeSet[protrusion]{basicmath} % disable protrusion for tt fonts
}{}
\makeatletter
\@ifundefined{KOMAClassName}{% if non-KOMA class
  \IfFileExists{parskip.sty}{%
    \usepackage{parskip}
  }{% else
    \setlength{\parindent}{0pt}
    \setlength{\parskip}{6pt plus 2pt minus 1pt}}
}{% if KOMA class
  \KOMAoptions{parskip=half}}
\makeatother
\usepackage{xcolor}
\IfFileExists{xurl.sty}{\usepackage{xurl}}{} % add URL line breaks if available
\IfFileExists{bookmark.sty}{\usepackage{bookmark}}{\usepackage{hyperref}}
\hypersetup{
  pdftitle={Layering Network Graphs},
  pdfborder={0 0 0},
  breaklinks=true}
\urlstyle{same}  % don't use monospace font for urls
\newif\ifbibliography
\usepackage{color}
\usepackage{fancyvrb}
\newcommand{\VerbBar}{|}
\newcommand{\VERB}{\Verb[commandchars=\\\{\}]}
\DefineVerbatimEnvironment{Highlighting}{Verbatim}{commandchars=\\\{\}}
% Add ',fontsize=\small' for more characters per line
\usepackage{framed}
\definecolor{shadecolor}{RGB}{248,248,248}
\newenvironment{Shaded}{\begin{snugshade}}{\end{snugshade}}
\newcommand{\AlertTok}[1]{\textcolor[rgb]{0.94,0.16,0.16}{#1}}
\newcommand{\AnnotationTok}[1]{\textcolor[rgb]{0.56,0.35,0.01}{\textbf{\textit{#1}}}}
\newcommand{\AttributeTok}[1]{\textcolor[rgb]{0.77,0.63,0.00}{#1}}
\newcommand{\BaseNTok}[1]{\textcolor[rgb]{0.00,0.00,0.81}{#1}}
\newcommand{\BuiltInTok}[1]{#1}
\newcommand{\CharTok}[1]{\textcolor[rgb]{0.31,0.60,0.02}{#1}}
\newcommand{\CommentTok}[1]{\textcolor[rgb]{0.56,0.35,0.01}{\textit{#1}}}
\newcommand{\CommentVarTok}[1]{\textcolor[rgb]{0.56,0.35,0.01}{\textbf{\textit{#1}}}}
\newcommand{\ConstantTok}[1]{\textcolor[rgb]{0.00,0.00,0.00}{#1}}
\newcommand{\ControlFlowTok}[1]{\textcolor[rgb]{0.13,0.29,0.53}{\textbf{#1}}}
\newcommand{\DataTypeTok}[1]{\textcolor[rgb]{0.13,0.29,0.53}{#1}}
\newcommand{\DecValTok}[1]{\textcolor[rgb]{0.00,0.00,0.81}{#1}}
\newcommand{\DocumentationTok}[1]{\textcolor[rgb]{0.56,0.35,0.01}{\textbf{\textit{#1}}}}
\newcommand{\ErrorTok}[1]{\textcolor[rgb]{0.64,0.00,0.00}{\textbf{#1}}}
\newcommand{\ExtensionTok}[1]{#1}
\newcommand{\FloatTok}[1]{\textcolor[rgb]{0.00,0.00,0.81}{#1}}
\newcommand{\FunctionTok}[1]{\textcolor[rgb]{0.00,0.00,0.00}{#1}}
\newcommand{\ImportTok}[1]{#1}
\newcommand{\InformationTok}[1]{\textcolor[rgb]{0.56,0.35,0.01}{\textbf{\textit{#1}}}}
\newcommand{\KeywordTok}[1]{\textcolor[rgb]{0.13,0.29,0.53}{\textbf{#1}}}
\newcommand{\NormalTok}[1]{#1}
\newcommand{\OperatorTok}[1]{\textcolor[rgb]{0.81,0.36,0.00}{\textbf{#1}}}
\newcommand{\OtherTok}[1]{\textcolor[rgb]{0.56,0.35,0.01}{#1}}
\newcommand{\PreprocessorTok}[1]{\textcolor[rgb]{0.56,0.35,0.01}{\textit{#1}}}
\newcommand{\RegionMarkerTok}[1]{#1}
\newcommand{\SpecialCharTok}[1]{\textcolor[rgb]{0.00,0.00,0.00}{#1}}
\newcommand{\SpecialStringTok}[1]{\textcolor[rgb]{0.31,0.60,0.02}{#1}}
\newcommand{\StringTok}[1]{\textcolor[rgb]{0.31,0.60,0.02}{#1}}
\newcommand{\VariableTok}[1]{\textcolor[rgb]{0.00,0.00,0.00}{#1}}
\newcommand{\VerbatimStringTok}[1]{\textcolor[rgb]{0.31,0.60,0.02}{#1}}
\newcommand{\WarningTok}[1]{\textcolor[rgb]{0.56,0.35,0.01}{\textbf{\textit{#1}}}}
\usepackage{graphicx,grffile}
\makeatletter
\def\maxwidth{\ifdim\Gin@nat@width>\linewidth\linewidth\else\Gin@nat@width\fi}
\def\maxheight{\ifdim\Gin@nat@height>\textheight\textheight\else\Gin@nat@height\fi}
\makeatother
% Scale images if necessary, so that they will not overflow the page
% margins by default, and it is still possible to overwrite the defaults
% using explicit options in \includegraphics[width, height, ...]{}
\setkeys{Gin}{width=\maxwidth,height=\maxheight,keepaspectratio}
\setlength{\emergencystretch}{3em}  % prevent overfull lines
\providecommand{\tightlist}{%
  \setlength{\itemsep}{0pt}\setlength{\parskip}{0pt}}
\setcounter{secnumdepth}{-2}

% set default figure placement to htbp
\makeatletter
\def\fps@figure{htbp}
\makeatother


\title{Layering Network Graphs}
\date{}

\begin{document}
\frame{\titlepage}

\begin{frame}[fragile]

\begin{verbatim}
## Registered S3 methods overwritten by 'ggplot2':
##   method         from 
##   [.quosures     rlang
##   c.quosures     rlang
##   print.quosures rlang
\end{verbatim}

\begin{verbatim}
## Registered S3 method overwritten by 'rvest':
##   method            from
##   read_xml.response xml2
\end{verbatim}

\begin{verbatim}
## -- Attaching packages -------------------------------------- tidyverse 1.2.1 --
\end{verbatim}

\begin{verbatim}
## v ggplot2 3.1.1       v purrr   0.3.2  
## v tibble  2.1.1       v dplyr   0.8.0.1
## v tidyr   0.8.3       v stringr 1.4.0  
## v readr   1.3.1       v forcats 0.4.0
\end{verbatim}

\begin{verbatim}
## -- Conflicts ----------------------------------------- tidyverse_conflicts() --
## x dplyr::filter()  masks stats::filter()
## x purrr::is_null() masks testthat::is_null()
## x dplyr::lag()     masks stats::lag()
## x dplyr::matches() masks testthat::matches()
\end{verbatim}

\begin{verbatim}
## 
## Attaching package: 'tidygraph'
\end{verbatim}

\begin{verbatim}
## The following object is masked from 'package:stats':
## 
##     filter
\end{verbatim}

\begin{verbatim}
## The following object is masked from 'package:testthat':
## 
##     matches
\end{verbatim}

\begin{verbatim}
## Parsed with column specification:
## cols(
##   id = col_character(),
##   s_id = col_logical(),
##   timestamp = col_datetime(format = ""),
##   subject_code = col_character(),
##   subject = col_character(),
##   firstname = col_character(),
##   lastname = col_character(),
##   role = col_character(),
##   row_status = col_double(),
##   session_id = col_double(),
##   event = col_character(),
##   handle = col_character(),
##   data = col_character(),
##   content_pk1 = col_double()
## )
\end{verbatim}

\begin{verbatim}
## Warning: 304065 parsing failures.
##  row  col           expected   actual                                            file
## 5711 s_id 1/0/T/F/TRUE/FALSE 99919526 '~/Data/Psychology/aa Biopsychology 201860.csv'
## 5712 s_id 1/0/T/F/TRUE/FALSE 99919526 '~/Data/Psychology/aa Biopsychology 201860.csv'
## 5713 s_id 1/0/T/F/TRUE/FALSE 99919526 '~/Data/Psychology/aa Biopsychology 201860.csv'
## 5714 s_id 1/0/T/F/TRUE/FALSE 99919526 '~/Data/Psychology/aa Biopsychology 201860.csv'
## 5715 s_id 1/0/T/F/TRUE/FALSE 99919526 '~/Data/Psychology/aa Biopsychology 201860.csv'
## .... .... .................. ........ ...............................................
## See problems(...) for more details.
\end{verbatim}

\begin{verbatim}
## Warning: Duplicated column names deduplicated: 'pk1' => 'pk1_1' [40],
## 'dtcreated' => 'dtcreated_1' [46], 'dtmodified' => 'dtmodified_1' [47],
## 'start_date' => 'start_date_1' [61], 'end_date' => 'end_date_1' [62],
## 'allow_guest_ind' => 'allow_guest_ind_1' [76], 'available_ind' =>
## 'available_ind_1' [80], 'allow_observer_ind' => 'allow_observer_ind_1' [81]
\end{verbatim}

\begin{verbatim}
## Parsed with column specification:
## cols(
##   .default = col_character(),
##   pk1 = col_double(),
##   dtcreated = col_datetime(format = ""),
##   dtmodified = col_datetime(format = ""),
##   position = col_double(),
##   offline_name = col_logical(),
##   offline_path = col_logical(),
##   start_date = col_logical(),
##   crsmain_pk1 = col_double(),
##   end_date = col_logical(),
##   description = col_logical(),
##   parent_pk1 = col_double(),
##   data_version = col_double(),
##   link_ref = col_logical(),
##   folder_type = col_logical(),
##   copy_from_pk1 = col_logical(),
##   pk1_1 = col_double(),
##   buttonstyles_pk1 = col_double(),
##   cartridge_pk1 = col_logical(),
##   classifications_pk1 = col_double(),
##   data_src_pk1 = col_double()
##   # ... with 26 more columns
## )
\end{verbatim}

\begin{verbatim}
## See spec(...) for full column specifications.
\end{verbatim}

\begin{verbatim}
## Joining, by = c("id", "title")
## Joining, by = c("id", "title")
\end{verbatim}

\end{frame}

\begin{frame}[fragile]{Base Network}
\protect\hypertarget{base-network}{}

Original network. This shows the connections between different parts of
the subject site.

\begin{verbatim}
## Warning in if (class(attr) == "list") {: the condition has length > 1 and
## only the first element will be used
\end{verbatim}

\includegraphics{layering_networks_files/figure-beamer/draw_base_graph-1.pdf}

\end{frame}

\begin{frame}{R Markdown}
\protect\hypertarget{r-markdown}{}

This is an R Markdown presentation. Markdown is a simple formatting
syntax for authoring HTML, PDF, and MS Word documents. For more details
on using R Markdown see \url{http://rmarkdown.rstudio.com}.

When you click the \textbf{Knit} button a document will be generated
that includes both content as well as the output of any embedded R code
chunks within the document.

\end{frame}

\begin{frame}{Slide with Bullets}
\protect\hypertarget{slide-with-bullets}{}

\begin{itemize}
\tightlist
\item
  Bullet 1
\item
  Bullet 2
\item
  Bullet 3
\end{itemize}

\end{frame}

\begin{frame}[fragile]{Slide with R Output}
\protect\hypertarget{slide-with-r-output}{}

\begin{Shaded}
\begin{Highlighting}[]
\KeywordTok{summary}\NormalTok{(cars)}
\end{Highlighting}
\end{Shaded}

\begin{verbatim}
##      speed           dist       
##  Min.   : 4.0   Min.   :  2.00  
##  1st Qu.:12.0   1st Qu.: 26.00  
##  Median :15.0   Median : 36.00  
##  Mean   :15.4   Mean   : 42.98  
##  3rd Qu.:19.0   3rd Qu.: 56.00  
##  Max.   :25.0   Max.   :120.00
\end{verbatim}

\end{frame}

\begin{frame}{Slide with Plot}
\protect\hypertarget{slide-with-plot}{}

\includegraphics{layering_networks_files/figure-beamer/pressure-1.pdf}

\end{frame}

\end{document}
